\documentclass[11pt,english]{IEEEtran}
\usepackage{mhchem}
\usepackage{siunitx}
\usepackage{physics}
\usepackage{amsmath}
\usepackage{cuted}
\usepackage{graphicx}
\graphicspath{ {images/} }

\author{
	Lee, Seungsup
	\and
	Miller, Dory
	\and
	Payant, Andrew
	\and
	Powers-Luhn, Justin
	\and
	Zhang, Fan
}
\title{Nuclear Reactor Theory Project \#1\\Group \#3}


\begin{document}
	\pagenumbering{gobble}
	\maketitle
	\newpage
	\pagenumbering{arabic}

	\begin{abstract}

	\end{abstract}
	
	\section{Introduction \& Background}
	Proving the capabilities and safety of a reactor design requires effective modeling of the neutron flux in the core (expressed in equation \ref{eqn:transport}). For real cores, however, this is impossible, and must be first simplified, then discretized to provide the solution for a representative mesh. 

	For this project we have analyzed a simplified, monoenergetic, non-multiplying medium in one dimension. The flux originates from a single source at $x=0$ with a strength of $S=\SI{10E8}{\per\second}$. These assumptions simplify the transport equation to that presented in equation \ref{eqn:simplified_diffusion}.

	In the following sections, we will first describe the terms in equation \ref{eqn:simplified_diffusion}, then provide both an analytical and a discrete solution. We will also provide an analysis of the accuracy of the analysis as a function of the number of nodes. Finally, we will analzye the solution of different coordinate systems on our solution.

	\begin{strip}
	\begin{multline}
		\pdv{n}{t} + v \hat{\Omega} \cdot \nabla n + v \Sigma_t n \left( \mathbf{r}, E^\prime, \hat{\Omega}, t \right) = \\ \int_{4\pi} d \hat{\Omega} ^\prime \int_0^{\infty} dE ^\prime v^\prime \Sigma_s\left( E ^\prime \rightarrow E, \hat{\Omega} ^\prime \rightarrow \hat{\Omega} \right) n\left( \mathbf{r}, E ^\prime, \hat{\Omega} ^\prime, t \right) + s\left( \mathbf{r}, E, \hat{\Omega}, t \right)
		\label{eqn:transport}
	\end{multline}
	\end{strip}

	\begin{equation}
		-D_m \dv[2]{\phi}{x} + \Sigma^m_a \phi = 
		\begin{cases}
			S & \left(x=0\right) \\
			0 & \left(x>0\right)
		\end{cases}
		\label{eqn:simplified_diffusion}
	\end{equation}

	\begin{table*}
		\begin{center}
		\begin{tabular}{ c c c c c }
			\hline
			\textit{Material} & $ \Sigma_{tr}(\si{cm^{-1}}) $ & $ \Sigma_a (\si{cm^{-1}}) $ & $ \nu \Sigma_f (\si{cm^{-1}}) $ & \textit{Relative Absorption} \\
			\hline
			\ce{H} & \num{1.79e-2} & \num{8.08e-3} & 0 & \num{0.053} \\
			\ce{O} & \num{7.16e-3} & \num{4.90e-6} & \num{0} & \num{0}\\
			\ce{Zr} & \num{2.91e-3} & \num{7.01e-4} & \num{0} & \num{0.005} \\
			\ce{Fe} & \num{9.46e-4} & \num{3.99e-3} & \num{0} & \num{0.026} \\
			\ce{^{235}U} & \num{3.08e-4} & \num{9.24e-2} & \num{0.145} & \num{0.602} \\
			\ce{^{238}U} &\num{6.95e-3} & \num{1.39e-2} & \num{1.20e-2} & \num{0.091} \\
			\ce{^{10}B} & \num{8.77e-6} & \num{3.41e-2} & \num{0} & \num{0.223} \\
			\hline
			& \num{3.62e-2} & \num{0.1532} & \num{0.1570} & \num{1.000} \\
			\hline
		\end{tabular}
		\label{materials_table}
		\caption{Macroscopic Cross Sections}
		\end{center}
	\end{table*}

	\section{Methodology}
	The methodology goes here. \ref{materials_table}
	
	\section{Results}
	The results go here
	
	\section{Conclusions}
	The conclusions go here



\end{document}
