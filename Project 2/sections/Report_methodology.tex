\documentclass[../main.tex]{subfiles}

\begin{document}

Lorem ipsum dolor simet

\section{Methodology}
	Equation \ref{eqn:simplified_diffusion} is a simplified description of neutron diffusion through a finite medium, similar to a point source travelling through a shielding material to a detector. The flux, therefore, depends on the transport cross section. This is accounted for in the term $D_m$, which is related to the transport coefficient by $D_m = 3 \Sigma_{tr}^{-1}$. Values for $\Sigma_{tr}$ for typical reactor materials are found in table \ref{materials_table}. 

	
\subsection{Numerical Approximation} \label{ssec:numerical}
	It is rare to be faced with a design that allows for an analytical solution. Fortunately, numerical analysis methods exist that allow for approximation of the analytical solution. By dividing our hypothetical medium into discrete sections with nodes at the boundaries between these sections, it is possible to express the flux vector with equation \ref{eqn:linalg_flux}:
	\begin{equation}
		\opmat{A} \vec{\phi} = \opmat{F} \vec{\phi}
		\label{eqn:linalg_flux}
	\end{equation}
	where $\vec{\phi}$ is the flux at each node. The operator $\opmat{F}$ describes the source of the multiplying medium.  Initial guesses of 1 were used forThe operator $\opmat{A}$ can be derived using the two-node approximation of the  second derivative formula:
	
	\begin{equation*}
		\left. \pdv[2]{\phi}{x} \right|_i \approx \frac{\phi_{i-1} - 2\phi_i + \phi_{i+1}}{\Delta x^2}
	\end{equation*}

	For non-edge nodes and the far right node, the derivation can be performed as follows:
	\begin{align*}
		-D_m \dv[2]{\phi}{x} &+ \Sigma_a \phi = \frac{1}{k} \nu \Sigma^m_f \phi \\
		-D_m \int_{x_i - \frac{\Delta x}{2}}^{x_i + \frac{\Delta x}{2}} \dv[2]{\phi}{x} dx &+ 
			\int_{x_i - \frac{\Delta x}{2}}^{x_i + \frac{\Delta x}{2}} \Sigma_a \phi dx = \int_{x_i - \frac{\Delta x}{2}}^{x_i + \frac{\Delta x}{2}} \dv[2] \frac{1}{k} \nu \Sigma^m_f \phi_i dx\\
		-D_m \left. \dv{\phi}{x} \right|_{x_i - \frac{\Delta x}{2}}^{x_i + \frac{\Delta x}{2}} &+ 
			\int_{x_i - \frac{\Delta x}{2}}^{x_i + \frac{\Delta x}{2}} \Sigma_a \phi dx =  \frac{1}{k} \nu \Sigma^m_f \phi_i \int_{x_i - \frac{\Delta x}{2}}^{x_i + \frac{\Delta x}{2}} \dv[2] dx \\
		-D_m \left( \frac{\phi_{i+1} - \phi_i}{\Delta x} - \frac{\phi_{i} - \phi_{i-1}}{\Delta x} \right) &+
			\int_{x_i - \frac{\Delta x}{2}}^{x_i + \frac{\Delta x}{2}} \Sigma_a \phi dx = \frac{1}{k} \nu \Sigma^m_f \phi_i \left[ \left( x_i + \frac{\Delta x}{2} \right) - \left( x_i - \frac{\Delta x}{2} \right) \right] \\
		-D_m \left( \frac{\phi_{i+1} - \phi_i}{\Delta x} - \frac{\phi_{i} - \phi_{i-1}}{\Delta x} \right) &+
			\Sigma_a \phi_i \Delta x = \frac{1}{k} \nu \Sigma^m_f \phi_i \Delta x \\
		\intertext{Dividing each term by $\Delta x$ gives:}
		\left( \frac{-D_m}{\Delta x^2} \right) \phi_{i-1} + \left( \frac{2D_m}{\Delta x^2} + \Sigma_a \right)\phi_i + \left( \frac{-D_m}{\Delta x^2} \right) \phi_{i+1} = \frac{1}{k} \nu \Sigma^m_f \phi_i
	\end{align*}
	From this we see that our operator $\opmat{A}$ will have non-corner terms and right bottom corner terms:
	**
	\begin{bmatrix}[1.5]
		\hdotsfor{5} \\
		\frac{-D_m}{\Delta x^2} & \frac{2D_m}{\Delta x^2} + \Sigma_a & \frac{-D_m}{\Delta x^2} & 0 & \dots \\
		0 & \frac{-D_m}{\Delta x^2} & \frac{2D_m}{\Delta x^2} + \Sigma_a & \frac{-D_m}{\Delta x^2} & \dots \\
		& & \ddots & & \\
		\dots & 0 & \frac{-D_m}{\Delta x^2} & \frac{2D_m}{\Delta x^2} + \Sigma_a & \frac{-D_m}{\Delta x^2} \\
	\end{bmatrix}
	**
	We then solve for our top row using the boundary condition in equation \ref{eqn:leftboundary}: 
	\begin{align*}
		-D_m \frac{\phi_1 - \phi_0}{\Delta x} + D_m \left. \dv{\phi}{x} \right|_0 + \int_0^{\Delta x / 2} 
			\Sigma_a \phi dx =   \frac{1}{k} \nu \Sigma^m_f \phi \\
		-D_m \frac{\phi_1 - \phi_0}{\Delta x} +  \int_0^{\Delta x / 2} 
			\Sigma_a \phi dx = \int_0^{\Delta x / 2}  \frac{1}{k} \nu \Sigma^m_f \phi dx \\
		-D_m \frac{\phi_1 - \phi_0}{\Delta x} + \Sigma_a \phi_0 \int_0^{\Delta x / 2}
			dx = \frac{1}{k} \nu \Sigma^m_f \phi0i \int_0^{\Delta x / 2} dx \\
			-D_m \frac{\phi_1 - \phi_0}{\Delta x} + \Sigma_a \phi_0  \frac{\Delta x}{2} = \frac{1}{k} \nu \Sigma^m_f \phi_0 \frac{\Delta x}{2} \\
	\intertext{Divide by $\Delta x$ on both sides:}
		-D_m \frac{\phi_1 - \phi_0}{\Delta x^2}  + \frac{1}{2} \Sigma_a \phi_0 =  \frac{1}{k} \nu \Sigma^m_f \phi_0 \frac{1}{2} \\
		\frac{-D_m}{\Delta x^2} \phi_1 + \left( \frac{D}{\Delta x^2} + \frac{1}{2} \Sigma_a \right) \phi_0 = \frac{1}{2k} \nu \Sigma^m_f \phi_0
	\end{align*}
	This gives a final matrix $\opmat{A}$ (for N nodes):
	\[
	\begin{bmatrix}[1.5]
		\frac{D_m}{\Delta x^2} + \frac{1}{2}\Sigma_a & - \frac{-D_m}{\Delta x^2} & 0 & 0 & \dots & 0\\
		- \frac{-D_m}{\Delta x^2} & \frac{2D_m}{\Delta x^2} + \Sigma_a & - \frac{-D_m}{\Delta x^2} & 0 & \dots & 0 \\
		0 & - \frac{-D_m}{\Delta x^2} & \frac{2D_m}{\Delta x^2} + \Sigma_a & - \frac{-D_m}{\Delta x^2} & \dots & 0 \\
		0 & 0 & - \frac{-D_m}{\Delta x^2} & \frac{2D}{\Delta x^2} + \Sigma_a & \dots & 0 \\
		\dots & \dots & \dots \dots & \frac{D_m}{\Delta x^2} + \frac{1}{2}\Sigma_a & - \frac{-D_m}{\Delta x^2} \\
		0 & \dots & 0 & 0 - \frac{-D_m}{\Delta x^2} & \frac{2 D_m}{\Delta x^2} + \frac{1}{2}\Sigma_a
	\end{bmatrix}
	\]


This gives a final matrix $\opmat{F}$ (for N nodes):
	\[
	\begin{bmatrix}[1.5]
		\frac{1}{2k} \nu \Sigma_f \phi_0  \\
		\frac{1}{k} \nu \Sigma_f \phi_1  \\
		\frac{1}{k} \nu \Sigma_f \phi_2  \\
		\frac{1}{k} \nu \Sigma_f \phi_3  \\
		\dots \\
		\frac{1}{k} \nu \Sigma_f \phi_{N-1}
	\end{bmatrix}
	\]
	
A similar method was used to find matrix $\opmat{A}$ for cylindrical and spherical coordinates.  However the center averaged method was used to maneuver $\phi$.

For cylindrical coordinates, the following equation was used to derive $\opmat{A}$:

\begin{equation*}
		-D_m \frac{d^2 \phi}{d r^2} + \frac{1}{r} \frac{d \phi}{dx} + \Sigma_a \phi =   \frac{1}{k} \nu \Sigma^m_f \phi
	\end{equation*}
	
The resulting matrix $\opmat{A}$ (for N nodes) is:

\[
	\begin{bmatrix}[1.5]
		\frac{1}{2} \Sigma_a + \frac{D_m}{(2i-1)\Delta r^2} +\frac{D_m}{r^2} & -\frac{D_m}{(2i-1)\Delta r^2} - \frac{D_m}{r^2} & 0 & \dots & 0 \\
		\frac{-D_m}{\Delta r^2} \left( 1 + \frac{1}{2i-1} \right) & \frac{2D_m}{\Delta r^2} + \Sigma_a & - \frac{-D_m}{\Delta r^2} \left( 1 - \frac{1}{2i-1} \right) & \dots & 0 \\
		0 & - \frac{-D_m}{\Delta r^2} \left( 1 - \frac{1}{2i-1} \right) & \frac{2D_m}{\Delta r^2} + \Sigma_a & \frac{-D_m}{\Delta r^2} \left( 1 - \frac{1}{2i-1} \right) & 0\\
		\dots & \dots \dots & \frac{2D_m}{\Delta r^2} + \Sigma_a & 0 \\
		0 & 0 & 0 & -\frac{-D_m}{\Delta r^2} \left( 1 - \frac{1}{2i-1} \right) & \frac{2D_m}{\Delta r^2} + \Sigma_a
	\end{bmatrix}
	\]

For spherical coordinates, the following equation was used to derive $\opmat{A}$:

\begin{equation*}
		-D_m \frac{d^2 \phi}{d r^2} + \frac{2}{r} \frac{d \phi}{dx} + \Sigma_a \phi =   \frac{1}{k} \nu \Sigma^m_f \phi
	\end{equation*}

The resulting matrix $\opmat{A}$ (for N nodes) is:

\[
	\begin{bmatrix}[1.5]
		\frac{1}{2} \Sigma_a + \frac{2 D_m}{(2i-1)\Delta r^2} +\frac{D_m}{r^2} & -\frac{D_m}{(2i-1)\Delta r^2} - \frac{2 D_m}{r^2} & 0 & \dots & 0 \\
		\frac{-D_m}{\Delta r^2} \left( 1 + \frac{2}{2i-1} \right) & \frac{2D_m}{\Delta r^2} + \Sigma_a & - \frac{-D_m}{\Delta r^2} \left( 1 - \frac{2}{2i-1} \right) & \dots & 0 \\
		0 & \frac{-D_m}{\Delta r^2} \left( 1 - \frac{2}{2i-1} \right) & \frac{2D_m}{\Delta r^2} + \Sigma_a & \frac{-D_m}{\Delta r^2} \left( 1 - \frac{2}{2i-1} \right) & 0\\
		\dots & \dots \dots & \frac{2D_m}{\Delta r^2} + \Sigma_a & 0 \\
		0 & 0 & 0 & -\frac{-D_m}{\Delta r^2} \left( 1 - \frac{2}{2i-1} \right) & \frac{2D_m}{\Delta r^2} + \Sigma_a
	\end{bmatrix}
	\]


\end{document}
