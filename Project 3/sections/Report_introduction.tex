\documentclass[../main.tex]{subfiles}

\begin{document}

Proving the capabilities and safety of a reactor design requires effective modeling of the neutron flux in the core (expressed in equation \ref{eqn:transport}). For real cores, however, this is impossible, and must be first simplified, then discretized to provide the solution for a representative mesh. Certain simplifications can be applied to help make the problem easier to solve, since it allows the time dependent terms to be removed.

1) Isotropic assumption: ignoring the direction of the incoming neutron. This effectively drops all terms involving Ωˆ
2) Discretized energies:  Neutrons ar placed into energy bins or groups
3) There is an interface between the moderator and reflector material.
4) Steady-state assumption: assuming that the system has been in this state for a long period and that no transients occur
5) There is no flux outside of the reflector.
6) All neutrons are born fast.



	For this project we have analyzed a cylindrical fuel pin surrounded by a cylindrical reflector.  The previous assumptions transform the transport equation to that presented in equation \ref{eqn:multi_group}.

	In the following sections, we will first describe the terms in equation \ref{eqn:multi_group}, then provide an analytical solution. The flux will also be analyzed in the cylindrical system using a programmed Fortran code with a chosen number of nodes and groups. We will also provide an analysis of the accuracy of the analysis as a function of the number of nodes and number of groups. 

	
	%\begin{strip}
	\begin{equation}
		\pdv{n}{t} + v \hat{\Omega} \cdot \nabla n + v \Sigma_t n \left( \mathbf{r}, E^\prime, \hat{\Omega}, t \right) = \\ \int_{4\pi} d \hat{\Omega} ^\prime \int_0^{\infty} dE ^\prime v^\prime \Sigma_s\left( E ^\prime \rightarrow E, \hat{\Omega} ^\prime \rightarrow \hat{\Omega} \right) n\left( \mathbf{r}, E ^\prime, \hat{\Omega} ^\prime, t \right) + s\left( \mathbf{r}, E, \hat{\Omega}, t \right)
		\label{eqn:transport}
	\end{equation}
	%\end{strip}
	
	%\begin{strip}
	\begin{equation}
		- \nabla \cdot D_{g} \nabla \phi_{g}+ v \Sigma_{Rg} \phi_g = \sum_{g^'=1}^{g-1} = \Sigma_{sg^'g} \phi_{g^'} + \frac{1}{k} \chi_g \sum_{g^'=1}^{G} \nu_{g^'} \Sigma_{tg^'} \phi_{g^'} 
		\label{eqn:multi_group}
	\end{equation}
	%\end{strip}
  
  \begin{equation}
		-D_m \dv[2]{\phi}{x} + \Sigma^m_a \phi = \frac{1}{k} \nu \Sigma^m_f \phi
		\label{eqn:simplified_diffusion}
	\end{equation}
	
The following eigenvector problem must be solved:
 \begin{equation}
		\opmat{A} \vec{b} = \opmat{F} \vec{b}
		\label{eqn:inverted_matrix}
	\end{equation}
  
  	\begin{table*}
		\begin{center}
		\begin{tabular}{ c c c }
			\hline
			\textit{Group Constant} & \textit{Group 1} & \textit{Group 2} \\
			\hline
			\ce{\nu \Sigma_f} & \num{0.008476} & \num{0.18514} \\
			\ce{\Sigma_f} & \num{0.003320} & \num{0.07537} \\
			\ce{\Sigma_a} & \num{0.01207} & \num{0.1210} \\
			\ce{D} & \num{1.2627} & \num{0.3543}  \\
			\ce{\Sigma_R} & \num{0.02627} & \num{0.1210} \\
			\hline
		\end{tabular}
		\label{2GXS__core_table}
		\caption{Reactor Core Cross Sections for Two Groups}
		\end{center}
	\end{table*}

\begin{table*}
		\begin{center}
		\begin{tabular}{ c c c }
			\hline
			\textit{Group Constant} & \textit{Group 1} & \textit{Group 2} \\
			\hline
			\ce{\Sigma_a} & \num{0.0004} & \num{0.0197} \\
			\ce{D} & \num{1.13} & \num{0.16}  \\
			\ce{\Sigma_R = \Sigma_{s12}} & \num{0.0494} & \ce{--} \\
			\hline
		\end{tabular}
		\label{2GXS_reflector_table}
		\caption{Water Reflector Cross Sections for Two Groups}
		\end{center}
	\end{table*}
	
\begin{table*}
		\begin{center}
		\begin{tabular}{ c c c c c}
			\hline
			\textit{Group Constant} & \textit{Group 1} & \textit{Group 2} & \textit{Group 3} & \textit{Group 4}\\
			\hline
			\ce{\nu \Sigma_f} & \num{0.009572} & \num{0.001193} & \num{0.01768} & \num{0.18514}\\
			\ce{\Sigma_f} & \num{0.003378} & \num{0.0004850} & \num{0.006970} & \num{0.07527}\\
			\ce{\Sigma_a} & \num{0.004946} & \num{0.002840} & \num{0.03053} & \num{0.1210}\\
			\ce{D} & \num{2.1623} & \num{1.0867} & \num{0.6318} & \num{0.3543}\\
			\ce{\Sigma_R} & \num{0.08785} & \num{0.06124} & \num{0.09506} & \num{0.1210}\\
			\hline
		\end{tabular}
		\label{4GXS__core_table}
		\caption{Reactor Core Cross Sections for Four Groups}
		\end{center}
	\end{table*}
	
\begin{table*}
		\begin{center}
		\begin{tabular}{ c c c }
			\hline
			\textit{Group Constant} & \textit{Group 1} & \textit{Group 2} & \textit{Group 3} & \textit{Group 4}\\
			\hline
			\ce{\Sigma_a} & \num{0.00051} & \num{0.00354} & \num{0.01581} & \num{0.04637}\\
			\ce{\Sigma_{tr}} & \num{0.20608} & \num{0.60215} & \num{0.56830} & \num{1.21110}\\
			\ce{\Sigma_R} & \num{0.08785} & \num{0.06124} & \num{0.09506} & \num{0.1210}\\
			\hline
		\end{tabular}
		\label{4GXS__reflector_table}
		\caption{Water Reflector Cross Sections for Four Groups}
		\end{center}
	\end{table*}

\end{document}
