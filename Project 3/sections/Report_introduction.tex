\documentclass[../main.tex]{subfiles}

\begin{document}

Proving the capabilities and safety of a reactor design requires effective modeling of the neutron flux in the core (expressed in equation \ref{eqn:transport}). For real cores, however, this is impossible, and must be first simplified, then discretized to provide the solution for a representative mesh. Certain simplifications can be applied to help make the problem easier to solve, since it allows the time dependent terms to be removed.

1) Isotropic assumption: ignoring the direction of the incoming neutron. This effectively drops all terms involving Ωˆ
2) Discretized energies:  Neutrons ar placed into energy bins
3) Interface between the moderator and reflector material.
4) Steady-state assumption: assuming that the system has been in this state for a long period and that no transients occur.



	For this project we have analyzed a cylindrical fuel pin surrounded by a cylindrical reflector.  The previous assumptions simplify the transport equation to that presented in equation \ref{eqn:multi_group}.

	In the following sections, we will first describe the terms in equation \ref{eqn:multi_group}, then provide an analytical solution. The flux will also be analyzed in the cylindrical system using a programmed coded with a chosen number of nodes and groups. We will also provide an analysis of the accuracy of the analysis as a function of the number of nodes and number of groups. 

	
	%\begin{strip}
	\begin{equation}
		\pdv{n}{t} + v \hat{\Omega} \cdot \nabla n + v \Sigma_t n \left( \mathbf{r}, E^\prime, \hat{\Omega}, t \right) = \\ \int_{4\pi} d \hat{\Omega} ^\prime \int_0^{\infty} dE ^\prime v^\prime \Sigma_s\left( E ^\prime \rightarrow E, \hat{\Omega} ^\prime \rightarrow \hat{\Omega} \right) n\left( \mathbf{r}, E ^\prime, \hat{\Omega} ^\prime, t \right) + s\left( \mathbf{r}, E, \hat{\Omega}, t \right)
		\label{eqn:transport}
	\end{equation}
	%\end{strip}
	
	%\begin{strip}
	\begin{equation}
		- \nabla \cdot D_{g} \nabla \phi_{g}+ v \Sigma_{Rg} \phi_g = \sum_{g^'=1}^{g-1} = \Sigma_{sg^'g} \phi_{g^'} + \frac{1}{k} \chi_g \sum_{g^'=1}^{G} \nu_{g^'} \Sigma_{tg^'} \phi_{g^'} 
		\label{eqn:multi_group}
	\end{equation}
	%\end{strip}
  
  \begin{equation}
		-D_m \dv[2]{\phi}{x} + \Sigma^m_a \phi = \frac{1}{k} \nu \Sigma^m_f \phi
		\label{eqn:simplified_diffusion}
	\end{equation}
	
The following eigenvector problem must be solved:
 \begin{equation}
		\opmat{A} \vec{b} = \opmat{F} \vec{b}
		\label{eqn:inverted_matrix}
	\end{equation}
  
  	\begin{table*}
		\begin{center}
		\begin{tabular}{ c c c c c }
			\hline
			\textit{Material} & $ \Sigma_{tr}(\si{cm^{-1}}) $ & $ \Sigma_a (\si{cm^{-1}}) $ & $ \nu \Sigma_f (\si{cm^{-1}}) $ & \textit{Relative Absorption} \\
			\hline
			\ce{H} & \num{1.79e-2} & \num{8.08e-3} & 0 & \num{0.053} \\
			\ce{O} & \num{7.16e-3} & \num{4.90e-6} & \num{0} & \num{0}\\
			\ce{Zr} & \num{2.91e-3} & \num{7.01e-4} & \num{0} & \num{0.005} \\
			\ce{Fe} & \num{9.46e-4} & \num{3.99e-3} & \num{0} & \num{0.026} \\
			\ce{^{235}U} & \num{3.08e-4} & \num{9.24e-2} & \num{0.145} & \num{0.602} \\
			\ce{^{238}U} &\num{6.95e-3} & \num{1.39e-2} & \num{1.20e-2} & \num{0.091} \\
			\ce{^{10}B} & \num{8.77e-6} & \num{3.41e-2} & \num{0} & \num{0.223} \\
			\hline
			& \num{3.62e-2} & \num{0.1532} & \num{0.1570} & \num{1.000} \\
			\hline
		\end{tabular}
		\label{materials_table}
		\caption{Macroscopic Cross Sections}
		\end{center}
	\end{table*}



\end{document}
