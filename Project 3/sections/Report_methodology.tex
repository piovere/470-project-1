\documentclass[../main.tex]{subfiles}

\begin{document}

	Equation \ref{eqn:multi_group} is a general description of multi-group neutron diffusion through a reactor core and reflector. The flux depends on the transport cross section, which is accounted for in the term D.  The relationship is defined by $D_m = 3 \Sigma_{tr}^{-1}$. Values for $\Sigma_{tr}$ for the reactor core and reflector are found in table \ref{2GXS_core_table}, \ref{2GXS_reflector_table}, \ref{4GXS_core_table}, \ref{4GXS_reflector_table}.
	
	In the two interpretations of multi-group theory, neutrons are divided into 2 groups (fast and slow) and 4 groups.  The finite-difference equation for each group must be derived separately because the source term for each group is different, as well as the constants used. 

	
\subsection{Numerical Approximation} \label{ssec:numerical}
	It is rare to be faced with a design that allows for an analytical solution. Fortunately, numerical analysis methods exist that allow for approximation of the analytical solution. By dividing our hypothetical medium into discrete sections with nodes at the boundaries between these sections and grouping neutrons into a set number of energy bins, it is possible to express the flux vector with equation \ref{eqn:linalg_flux}:
	\begin{equation}
		\opmat{A} \vec{\phi} = \opmat{F} \vec{\phi}
		\label{eqn:linalg_flux}
	\end{equation}
	where $\vec{\phi}$ is the flux at each node. The operator $\opmat{F}$ describes the source of the multiplying medium.  Initial guesses of 1 were used for both k and the source vector.  The flux of each group must be solved using the previous equation  \ref{eqn:linalg_flux}.  Do note that the super or subscript m and R correspond to the core and reflector, respectively.  The number in the subscript corresponds to the group number.  Also not that r and $r_R$ correspond to the radius of the core and reflector, respectively.  The initial radius of the core is denoted as W. 
	
	When using 2 groups, the operator $\opmat{A}$ can be derived from the following equations:
	
	\begin{equation*}
		- \nabla \cdot D_{1} \nabla \phi_{1}+ \Sigma_{R_1} \phi_1 = \frac{1}{k} [nu_1 \Sigma_{f_1} \phi_{1} + nu_2 \Sigma_{f_2} \phi_{2}] 
	\end{equation*}
	
	\begin{equation*}
		- \nabla \cdot D_{2} \nabla \phi_{2}+ \Sigma_{a_2} \phi_2 = \Sigma_{s_{12}} \phi_1
	\end{equation*}

Only the first group has a fission source, while other groups gain neutrons from the downscatter of faster groups.  THe flux of the previous group is then used in the calcuation of the flux for the following group(s).

The internal nodes of both the reactor core and reflector can be derived with the same process as Project 2 for cylindrical coordinates.  The following general equation was used to derive the internal nodes of the reflector and moderator  and initial boundary of moderator for the first group:

\begin{equation*}
		-D_m \frac{d^2 \phi}{d r^2} + \frac{1}{r} \frac{d \phi}{dx} + \Sigma_a \phi =   \frac{1}{k} \nu \Sigma^_mf_1 \phi
	\end{equation*}
	


The resulting contribution from the core to matrix $\opmat{A}$ (for N nodes) is:

\[
	\begin{bmatrix}[1.5]
		 \Sigma^m_R \delta r + \frac{2D^m_g}{\delta r} & -\frac{D^m_g}}{\delta r} & 0 & \dots & 0 \\
		-\frac{D^m_g}{\delta r} \left( 1 - \frac{1}{2i-1} \right) & \frac{2D^m_g}{\delta r} + \Sigma^m_R \delta r &  -\frac{D^m_g}{\delta r} \left( 1 + \frac{1}{2i-1} \right) & \dots & 0 \\
		0 &  -\frac{D^m_g}{\delta r} \left( 1 - \frac{1}{2i-1} \right) & \frac{2D^m_g}{\delta r} + \Sigma^m_R \delta r &  -\frac{D^m_g}{\delta r} \left( 1 + \frac{1}{2i-1} \right) & 0\\
		\dots & \dots \dots & \frac{2D^m_g}{\delta r} & \dots \\
		0 & 0 & 0 & -\frac{D^m_g}{\delta r} \left( 1 - \frac{1}{2i-1} \right) & \frac{2D^m_g}{\delta r} + \Sigma^m_R \delta r
	\end{bmatrix}
	\]
	
This part of the matrix $\opmat{A}$ is located at the top left of the entire matrix.
	
The resulting contribution from the reflector to matrix $\opmat{A}$ (for N nodes) is:

\[
	\begin{bmatrix}[1.5]
		-\frac{D^R_g}{\delta r_R} \left( 1 - \frac{1}{2i-1} \right) & \frac{2D^R_g}{\delta r_R} + \Sigma^R_a \delta r &  -\frac{D^R_g}{\delta r_R} \left( 1 + \frac{1}{2i-1} \right) & \dots & 0 \\
		0 &  -\frac{D^R_g}{\delta r_R} \left( 1 - \frac{1}{2i-1} \right) & \frac{2D^R_g}{\delta r_R} + \Sigma^R_a \delta r_R &  -\frac{D^R_g}{\delta r_R} \left( 1 + \frac{1}{2i-1} \right) & 0\\
		\dots & \dots \dots & \\frac{2D^R_g}{\delta r_R} + \Sigma^R_a \delta r_R  & \dots \\
		0 & 0 & 0 & -\frac{D^R_g}{\delta r_R} \left( 1 - \frac{1}{2i-1} \right) & \frac{2D^R_g}{\delta r_R} + \Sigma^R_a \delta r_R
	\end{bmatrix}
	\]

This part of the matrix $\opmat{A}$ is located at the bottom right of the entire matrix.

	
Connecting the reactor core and reflector is the interface node.  This means contributions from both materials must be accounted for at that point.  The following represents the interface node implemented in the matrix:


\[
	\begin{bmatrix}[1.5]
		 
		\dots & -\frac{D^m_g}}{\delta r} & frac{D^m_g}}{\delta r} + \frac{D^R_g}}{\delta r_R} + \frac{1}{2} \Sigma^m_R \delta r + \frac{1}{2} \Sigma^R_a \delta r_R & -\frac{D^R_g}}{\delta r_R}  & \dots \\
		
	\end{bmatrix}
	\]


	\begin{equation}
	\left( \frac{-D^m_1}{\delta r} + \frac{D^m_1}{2W} \right) \phi_{1,i-1} + \left( \frac{-D^m_1}{\delta r} + \frac{D^m_1}{2W} + \frac{\Sigma^m_{a,1} \delta r}{2} + \frac{D^R_1}{\delta r_R} + \frac{D^R_1}{2W} + \frac{\Sigma^R_{a,1} \delta r_R}{2} \right) \phi_{1,i} + \left( \frac{-D^R_1}{\delta r_R} + \frac{D^R_1}{2W} \right) \phi_{1,i+1}
	\end{equation}
	
The finite-difference equation for the reflector is solved with the same method, excluding the boundary condition.  The flux is forced to zero at the outside edge of the reflector.  The resulting left side of the finite-difference is:

	\begin{equation}
	\frac{-D_R}{\Delta r_R^2} \left( 1 + \frac{1}{2i-1} \right) \phi_{1,i-1} + \left( \frac{2D_R}{\Delta r_R^2} + \Sigma_{a,1} \right) \phi_{1,i} + \frac{-D_R}{\Delta r_R^2} \left( 1 - \frac{1}{2i-1} \right) \phi_{1,i+1}
	\end{equation}
	
Ther terms from this equation is implented in the matrix in the row between the core and reflector contributions.

Now let's take a look at the source terms.  All source terms are pushed to zero at the last node, since there is not flux outside of the reflector.  For the fast group, the source term  is the fission term.  This gives a final matrix $\opmat{F}$ (for N nodes):
	\[
	\begin{bmatrix}[1.5]
		\frac{1}{2k} \nu \Sigma^m_f  & 0 & 0 & 0 & \dots\\
		0 & \frac{1}{k} \nu \Sigma^m_f  & 0 & 0 & 0 & \dots \\
		0 & 0 & \frac{1}{k} \nu \Sigma^m_f  & 0 & 0 & \dots\\
		0 & 0 & 0 & \frac{1}{k} \nu \Sigma^m_f  & 0 & \dots \\
		\dots & \dots & \dots & \dots & \dots & \dots \\
		\dots & 0 & 0 & 0 & 0 & 0 \\
	\end{bmatrix}
	\]
	
For the slow group, the final matrix $\opmat{F}$ (for N nodes) is:

\[
	\begin{bmatrix}[1.5]
		\frac{1}{2} \Sigma^m_{s12} \phi_1  & 0 & 0 & 0 & \dots\\
		0 & \Sigma^m_{s12} \phi_1 & 0 & 0 & \dots & \dots \\
		0 & 0 & \Sigma^m_{s12} \phi_1 & 0 & 0 & \dots\\
		0 & \dots & 0 & \frac{1}{2} \Sigma^m_{s12} \phi_1 \delta r + \frac{1}{2} \Sigma^R_{s12} \phi_1 \delta r_R  & 0 & \dots \\
		\dots & 0 & 0 & 0 & \Sigma^R_{s12} \phi_1 & 0 \\
		\dots & 0 & 0 & 0 & \dots & \Sigma^R_{s12} \phi_1 \\
		\dots & \dots & \dots & \dots & \dots & \dots \\
		\dots & 0 & 0 & 0 & \dots & 0 \\
	\end{bmatrix}
	\]

The following equations were solved to find the flux for each of the four groups:

\begin{equation*}
		- \nabla \cdot D_{1} \nabla \phi_{1}+ \Sigma_{R_1} \phi_1 = \frac{1}{k} [nu_1 \Sigma_{f_1} \phi_{1} + nu_2 \Sigma_{f_2} \phi_{2}] 
	\end{equation*}
	
	\begin{equation*}
		- \nabla \cdot D_{2} \nabla \phi_{2}+ \Sigma_{a_2} \phi_2 = \Sigma_{s_{12}} \phi_1
	\end{equation*}
	
	\begin{equation*}
		- \nabla \cdot D_{3} \nabla \phi_{3}+ \Sigma_{a_3} \phi_3 = \Sigma_{s_{13}} \phi_1 + \Sigma_{s_{23}} \phi_2
	\end{equation*}
	
	\begin{equation*}
		- \nabla \cdot D_{4} \nabla \phi_{4}+ \Sigma_{a_4} \phi_4 = \Sigma_{s_{14}} \phi_1 + \Sigma_{s_{24}} \phi_2 + \Sigma_{s_{34}} \phi_3
	\end{equation*}
	
All the groups have a similar left side of their respective finite-difference equations, with the exception of the values of the cross sections and diffusion coefficients.  The main difference is in the source term for each group.  For the remaining groups, 3 and 4, the matrix $\opmat{F}$ (for N nodes) is:

Group 3:
\[
	\begin{bmatrix}[1.5]
		\frac{1}{2} \Sigma^m_{s13} \phi_1 + \frac{1}{2} \Sigma^m_{s23} \phi_2  & 0 & 0 & 0 & \dots\\
		0 & \Sigma^m_{s13} \phi_1 + \Sigma^m_{s23} \phi_2 & 0 & 0 & \dots & \dots \\
		0 & 0 & \Sigma^m_{s13} \phi_1 + \Sigma^m_{s23} \phi_2 & 0 & 0 & \dots\\
		0 & \dots & 0 & \frac{1}{2} \delta r \left( \Sigma^m_{s13} \phi_1 + \Sigma^m_{s23} \phi_2 \right) + \frac{1}{2} \delta r_R \left( \Sigma^R_{s13} \phi_1 + \Sigma^R_{s23} \phi_2 \right) & 0 & \dots \\
		\dots & 0 & 0 & 0 & \Sigma^R_{s13} \phi_1 + \Sigma^R_{s23} \phi_2  & 0 \\
		\dots & 0 & 0 & 0 & \dots & \Sigma^R_{s13} \phi_1 + \Sigma^R_{s23} \phi_2 \\
		\dots & \dots & \dots & \dots & \dots & \dots \\
		\dots & 0 & 0 & 0 & \dots & 0 \\
	\end{bmatrix}
	\]
	
Group 4:

\[
	\begin{bmatrix}[1.5]
		\frac{1}{2} \Sigma^m_{s14} \phi_1 + \frac{1}{2} \Sigma^m_{s24} \phi_2 + \frac{1}{2} \Sigma^m_{s34} \phi_3 & 0 & 0 & 0 & \dots\\
		0 & \Sigma^m_{s13} \phi_1 + \Sigma^m_{s24} \phi_2 + \Sigma^m_{s34} \phi_3 & 0 & 0 & \dots & \dots \\
		0 & 0 & \Sigma^m_{s14} \phi_1 + \Sigma^m_{s23} \phi_2 & 0 & 0 & \dots\\
		0 & \dots & 0 & \frac{1}{2} \delta r \left( \Sigma^m_{s14} \phi_1 + \Sigma^m_{s24} \phi_2 + \Sigma^m_{s34} \phi_3 \right) + \frac{1}{2} \delta r_R \left( \Sigma^R_{s14} \phi_1 + \Sigma^m_{s24} \phi_2 + \Sigma^R_{s34} \phi_3 \right) & 0 & \dots \\
		\dots & 0 & 0 & 0 & \Sigma^R_{s14} \phi_1 + \Sigma^m_{s24} \phi_2  + \Sigma^R_{s34} \phi_3 & 0 \\
		\dots & 0 & 0 & 0 & \dots & \Sigma^R_{s14} \phi_1 + \Sigma^m_{s24} \phi_2 + \Sigma^R_{s34} \phi_3 \\
		\dots & \dots & \dots & \dots & \dots & \dots \\
		\dots & 0 & 0 & 0 & \dots & 0 \\
	\end{bmatrix}
	\]
	
	


\end{document}
